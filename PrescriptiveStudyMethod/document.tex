	\documentclass[11pt,a4paper]{article}
	\usepackage[italian]{babel}
	\usepackage[utf8]{inputenc}
	
\begin{document}
\title{Metodo di studio prescrittivo a cascata.}
\author{Rombo}
\maketitle
\newpage

\section{Concetti preliminari}
 \subsection{Obbiettivo del documento. }
l'obbiettivo di questo documento è quello di fornire uno strumento o delle linee guide che possano essere d`aiuto per aumentare l'efficienza nelle ore di studio, in particolare si propongono delle strategie secondo modello prescrittivo a cascata per incrementare la qualità/ quantità delle nozioni imparate a parità di tempo investito nello studio.\\

\subsection{Obbiettivo Fondamentale.}
$\rightarrow$ Si definisce "Obbiettivo fondamentale" tutto quello che permette di discretizzare  l`intero esame in dei moduli più piccoli (esempio i pdf rilasciati dai professori) da studiare, l`intento è quello di trasformare un problema grande come lo studio di un esame in tanti piccoli problemi affrontabili e che diano la possibilità di valutare l`avanzamento di studio all'interno della materia.\\

\subsection{Tempo.}
$\rightarrow$ All'interno di questo metodo il tempo viene considerato come una risorsa tangibile, per tanto esso può essere in eccesso, in difetto ed accumulabile.\\
$\rightarrow$ l'unita temporale che verrà utilizzata all'interno del documento sarà l'ora, essa sarà composta da 45 minuti di "Sprint" ovvero una fase di sforzo intenso e da 15 minuti di "Cool Down" ovvero un momento di pausa all'interno della quale è consigliato abbandonare la postazione di studio a favore di un poco di movimento. \\
$\rightarrow$ Per quanto riguarda l'amministrazione del tempo vengono consigliati due modelli, il primo è il  modello arbitrario, il quale prevede che dopo 45 minuti di studio intenso arbitrariamente ci si ferma interrompendo il lavoro. Il secondo modello prevede invece di interrompere lo studio solo dopo il raggiungimento di un obbiettivo fondamentale, in questo caso si consiglia di conteggiare il tempo di "Sprint" fatto e di conseguenza il tempo di pausa accumulato. (Per esempio se io studiassi consecutivamente per due ore alla fine della seconda ho accumulato 30 minuti di pausa).\\
$\rightarrow$ Il tempo di lavoro giornaliero può essere deciso sulla base del carico di studio e delle previsioni fatte, in particolare si consiglia di tenere una linea massima di 7 ore al giorno e una linea minima di 5 ore al giorno. Si vuole far riflette di come 7 ore al giorno si possono vedere come 2,5 ore consecutive alla mattina (in quanto come già detto 2 ore consecutive di studio permettono di accumulare 30 minuti di pausa, il tempo di pausa verrà quindi sottratto alla terza ora della quale rimarranno quindi 30 minuti) e 3 ore di studio consecutive il pomeriggio.\\

\subsection{Lo studio come un problema.} 
$\rightarrow$ Per definizione problema è tutto ciò che per essere risolto impone l'assunzione di un adattamento do i un comportamento particolare, volto al raggiungimento di un obbiettivo. Lo studiare ricade pienamente in questa definizione e per tanto può essere visto come un problema.\\
Il modo migliore per risolvere un problema è suddividere il suddetto in una serie di piccole interazioni da eseguire fino al raggiungimento del obbiettivo. Anche se il modello suggerito è prescrittivo ed a cascata, questo non vieta di considerare come modello a cascata dei macro processi non iterativi (che quindi a conclusione ordinata dei suddetti si suppone aver raggiunto la soluzione del problema) e come modello a iterazioni la divisione iterativa di un macro processo.\\
Riassumendo noi stiamo dividendo un grande problema come quello dello studio in dei piccoli problemi, i quali a loro volta verranno divisi ulteriormente i dei problemi ancora più piccoli di nauta iterativa.\\ 

\subsection{Strumenti}
$\rightarrow$ questo metodo si serve, di un pc, un blocco di carta e delle penne di colore diverso con cui poter scrivere sul blocco di carta. i colori delle penne dovrebbero avere un significato, in particolare si consiglia di usare il colore nero per scrivere le cose "base" ovvero tutto quello che non ha bisogno di essere messo in risalto, in blu le cose importanti degli appunti (come per esempio concetti importanti, nomi o sigle]), in verde si scrivono gli accorgimenti ovvero quelle osservazioni personali che potrebbero aiutare a ricordare con facilità gli argomenti, in rosso Le cose molto importanti/ i titoli dei vari argomenti.\\
\newline

\section{Il metodo}
\subsection{Raccolta di risorse}
$\rightarrow$ Nella fase di raccolta delle risorse bisogna ricercare ed accumulare ordinatamente tutte le risorse (ogni tipo di risorsa come per esempio i documenti riguardanti la materia, ma anche il tempo disponibile, come anche in caso di necessità una persona che possa fornire aiuto).\\
$\rightarrow$Uno dei modi migliori per compiere questa fase è quello di avere quanto più possibile materiale utile allo studio in formato digitale, in modo da poter usufruire di tutti i vantaggi messi a disposizione dai programmi moderni oltre che poter recuperare volume nella area di lavoro. i file dovrebbero essere stoccati ed ordinati all`interno di una cartella posta in un posto sicuro. (Si definisce posto sicuro una cartella posta in una era dove normalmente non si fa del lavoro intensivo, un esempio di luogo non sicuro dove stoccare i documenti è il desktop del pc, perché la possibilità di eliminazione accidentale dei file è più alta, oltre che per il fatto che è una memoria maggiormente stressata rispetto ad altre, quindi è pure più alto il rischio di perdita naturale dei documenti a causa di usura, la cosa migliore quindi è possedere una memoria dedicata per custodire questo tipo di documenti più importanti).\\
I file all'interno di questa cartella dovrebbero essere poi suddivisi (creando ulteriori cartelle) in base alla importanza o in base al loro uso, in modo da ridurre il tempo di ricerca di una risorsa. Quando si studia una delle cose più importanti è mettersi nella condizione di perdere meno tempo possibile per fare delle azioni "Meccaniche" intese come azioni che non portano vantaggio per quanto riguarda l`assimilazione dei concetti.\\ 
$\rightarrow$ I nomi dei documenti e la loro funzione devono essere segnate in maniera ordinata (quindi di veloce lettura) sul blocco e tutte le altre risorse utili allo studio, particolare attenzione deve essere fatta per quanto riguarda il tempo, una pessima stima del tempo disponibile potrebbe provocare successivamente delle pessime valutazioni, aumentando così la probabilità di fare "Crouching" ovvero di dover svolgere una importante mole di lavoro in poco tempo.\\

\subsection{Analisi del materiale raccolto}
$\rightarrow$ Nella fase di analisi l`obbiettivo è quello di scrivere sempre sul blocco una stima del calendario, ovvero una stima del ritmo di studio che ci si aspetta di tenere in relazione dei dati raccolti, un esempio di ciò potrebbero essere i giorni che si pensa di studiare prima di essere pronti per la prova d'esame.\\
$\rightarrow$ Il documento che si andrà a scrivere sarà quindi la stima del lavoro richiesto per lo studio della materia, durante lo studio una parte fondamentale del lavoro sarà quella di segnarsi per ogni giorno lavorativo gli obbiettivi raggiunti, in modo da poter aggiornare la stima. In questo modo oltre che a prendere coscienza del effettivo rendimento (utile in caso di stime successive), si avrà anche un valido strumento per valutare di alleggerire il carico di studio qualora il nostro rendimento è superiore rispetto alle stime, oppure di aumentare il carico qualora ci si trovasse in difetto rispetto alla stima.\\

\subsection{Prendere coscienza degli argomenti}
$\rightarrow$ Nella fase di presa di coscienza degli argomenti l`obbiettivo è quello di leggere tutto il materiale raccolto, per ogni capitolo/obbiettivo di fare due tipi di schemi in contemporanea, il particolare è importante riassumere a pc ogni argomento letto in modo da poter avere della documentazione più agile da consultare nelle fasi successive.\\
$\rightarrow$ Il secondo schema richiesto è una mappa concettuale che sappia rappresentare in massimo una pagina del blocco di carta l`intero capitolo letto, al contrario del primo (per il quale non si pone limiti di spazio) deve essere il più piccolo possibile, in maniera tale da poter inquadrare al meglio gli elementi fondamentali del capitolo, inoltre uno schema di piccole dimensioni è più maneggevole, attitudine utile per le fasi successive.\\

\subsection{Macro visione}
$\rightarrow$ Nella fase di macro visione si fa l`operazione contraria della fase precedente, in particolare, se prima si è scesi nel dettaglio per ogni argomento ora si cerca di ottenere tramite le mappe concettuali fatte nella fase precedente una visione d`insieme di tutto l'esame. Le mappe concettuali fatte precedentemente devono essere composte e se possibile viste tutte in contemporanea, in maniera da poter apprezzare (magari aiutandosi con i riassunti) le relazioni che le legano.\\
$\rightarrow$ In questa fase una azione da non sottovalutare è quella di mettere in relazione i riassunti con le varie mappe concettuali, in maniera tale da potersi orientare con maggiore agilità tra entrambi i supporti.\\

\subsection{Curare l'esposizione}
$\rightarrow$ Una volta arrivati in questa fase si dovrebbe essere arrivati ad un livello di conoscenze sufficiente da potersi esercitare. Se l'esame è di tipo "scritto" si prendono gli esercizi collezionati durante la fase di raccolta del materiale e si svolgono finché non ci si sente agili nello svolgerli, oppure finché ci sono nuovi esercizi da fare. In caso di dubbi usare il materiale prodotto per risolvere il problema, se in più occasioni si verificasse la necessità di rileggere i documenti "originali", ovvero di documenti riassunti, questo è sintomo che il lavoro fatto non è valido allora si potrebbe prendere in considerazione di non dare più l'esame.\\
$\rightarrow$ Nel caso in cui l`esame fosse di tipo orale in questa fase bisogna curare l`esposizione degli argomenti, in particolare si consiglia di fare ordine tra i propri pensieri scrivendo le risposte alle domande più probabili che il professore potrebbe fare, confrontare poi il risultato con il materiale, valutare in che misura le risposte date si distaccano rispetto alle nozioni scritte. Se si dovessero distaccare minimamente allora si dovrebbe continuare a lavorare alla l`esposizione fino al raggiungimento al livello che si ritiene soddisfacente rispetto allo sforzo fatto nello studio della materia. Nel caso in cui si avverte che le risposte date di discostano in modo importante rispetto alla teoria, potrebbe essere il caso di rinunciare a dare l`esame.\\

\section{Conclusioni}
Come precedentemente detto questo metodo si propone di fornire uno strumento di aiuto allo studio, a seconda delle necessità queste linee guida dovrebbero essere poi personalizzate. L'obbiettivo finale del progetto è pure quello di fare in modo di poter affrontare lo studio con maggiore leggerezza, in quanto prendere coscienza del proprio progresso e del proprio stato dovrebbe essere un incentivo per uno studio quanto più possibile privo di ansie.\\
In particolare si consiglia al fine di uno studio più produttivo di prendersi i propri spazzi per rilassare la mente e per rigenerarla, trovare il tempo di fare almeno per due ore al giorno della attività fisica è un ottimo modo per rigenerarsi e per aumentare la quantità di energia.\\

	
\end{document}